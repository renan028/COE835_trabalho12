\section{Implementa��o}

Foram considerados dois casos: caso em que $K_p$ � desconehcido; e caso em que
todos os par�metros da planta s�o desconhecidos.

\subsection{Caso em que apenas $K_p$ � desconhecido}

Consideremos a planta descrita pela eq.~\ref{eq:P}. A equa��o ideal para o
controlador 2DOF poode ser computada pela equa��o do erro eq.~\ref{eq:erro}:

\begin{align*}
u^* &=  \theta_1^{*\intercal}\omega_1 + \theta_2^{*\intercal}\omega_2 +
\theta_3y + \theta_4r \\
\omega_1 &= \frac{u^*}{\Lambda} \\
\omega_2 &=  \frac{y}{\Lambda} \\
\left(I - \theta_1^{*\intercal}\Lambda^{-1}\right)u^* &=
\theta_2^{*\intercal}\Lambda^{-1}y + \theta_3 y + \theta_4 r 
\end{align*}

Temos que: 

\begin{align*}
\frac{1}{\Lambda(s)} &= \begin{bmatrix}
\frac{1}{(s+\lambda_f)} & 0\\
0 & \frac{1}{(s+\lambda_f)}
\end{bmatrix}
\end{align*}

e

\begin{align*}
M(s) &= \begin{bmatrix}
\frac{\lambda^2}{(s+\lambda)^2} & 0\\
0 & \frac{\lambda^2}{(s+\lambda)^2}
\end{bmatrix}
\end{align*}

Voltando � equa��o da planta:

\begin{align*}
y &= PK_pu \\
\left(I - \theta_1^{*\intercal}\Lambda^{-1}\right)y &= \left(I -
\theta_1^{*\intercal}\Lambda^{-1}\right)P\bar{u}\\
\end{align*}

Como $P(s)$ � diagonal, temos $AP = PA$ (comuta). E $y = y_m$, logo:

\begin{align*}
\left(I - \theta_1^{*\intercal}\Lambda^{-1}\right)y_m &= P\left(I -
\theta_1^{*\intercal}\Lambda^{-1}\right)\bar{u}\\
\left(I - \theta_1^{*\intercal}\Lambda^{-1}\right)M(s)r &= P(\theta_2^{*\intercal}\Lambda^{-1}y_m +
\theta_3 y_m + \theta_4 r)\\
\left(I - \theta_1^{*\intercal}\Lambda^{-1}\right)M(s)r &= P(\theta_2^{*\intercal}\Lambda^{-1}M(s)r
+ \theta_3 M(s)r + \theta_4 r)\\
\left(I - \theta_1^{*\intercal}\Lambda^{-1}\right)M(s)r &= P(\theta_2^{*\intercal}\Lambda^{-1}M(s)
+ \theta_3 M(s) + \theta_4)r\\
\left(I - \theta_1^{*\intercal}\Lambda^{-1}\right)M(s) &= P(\theta_2^{*\intercal}\Lambda^{-1}M(s)
+ \theta_3 M(s) + \theta_4)\\
\end{align*}

\begin{align*}
\left(\begin{bmatrix}
1 & 0\\ 0 & 1
\end{bmatrix}-\theta_1^{*\intercal}\begin{bmatrix}
\frac{1}{(s+\lambda_f)} & 0\\ 0 & \frac{1}{(s+\lambda_f)}
\end{bmatrix}\right)\begin{bmatrix}
\frac{\lambda^2}{(s+\lambda)^2} & 0\\
0 & \frac{\lambda^2}{(s+\lambda)^2}
\end{bmatrix} &= \\
\begin{bmatrix}
\frac{1}{s^2} & 0\\0 & \frac{1}{s^2}
\end{bmatrix} \left(\theta_2^{*\intercal}\begin{bmatrix}
\frac{1}{(s+\lambda_f)} & 0\\ 0 & \frac{1}{(s+\lambda_f)}
\end{bmatrix}\begin{bmatrix}
\frac{\lambda^2}{(s+\lambda)^2} & 0\\
0 & \frac{\lambda^2}{(s+\lambda)^2}
\end{bmatrix} + \theta_3\begin{bmatrix}
\frac{\lambda^2}{(s+\lambda)^2} & 0\\
0 & \frac{\lambda^2}{(s+\lambda)^2}
\end{bmatrix} + \theta_4\right) & 
\end{align*}

Como todas as matrizes s�o diagonais, vamos resolver para o primeiro termo:

\begin{align*}
\left(1-\frac{\theta_1}{(s+\lambda_f)}\right)\frac{\lambda^2}{(s+\lambda)^2} &=
\frac{1}{s^2}\left(\frac{\theta_2}{(s+\lambda_f)}\frac{\lambda^2}{(s+\lambda)^2}
+ \theta_3 \frac{\lambda^2}{(s+\lambda)^2} + \theta_4\right)
\end{align*}

\begin{align*}
(\lambda^2-\theta_4)s^3 +
(\lambda^2\lambda_f - \theta_1\lambda^2 - \theta_4(\lambda_f + 2\lambda))s^2 +
(-\theta_3\lambda^2 - \theta_4(2\lambda\lambda_f+\lambda^2))s +
(-\theta_2\lambda^2 - \theta_3\lambda^2\lambda_f - \theta_4\lambda^2\lambda_f)
&= 0
\end{align*}

Resolvendo para $\theta$, temos:

\begin{align*}
\theta_4 &= \lambda^2\\
\theta_1 &= -2\lambda \\ 
\theta_3 &= -(2\lambda\lambda_f + \lambda^2) \\
\theta_2 &= 2\lambda\lambda_f^2\\
\end{align*}
 
Desssa forma, a lei de controle � descrita como:

\begin{align}
u^* = K_p^{-1}\left[2\lambda\lambda_f^2\Lambda^{-1}y -
(\lambda^2+2\lambda\lambda_f)y + \lambda^2r\right] - 2\lambda\Lambda^{-1} u
\end{align}

Utilizando o algoritmo de Monopoli~\ref{eq:monopoli}, temos:
 
\begin{align}
u &= \Omega^\intercal\Theta + \zeta^\intercal\dot{\Theta} - 2\lambda\omega_1\\
\nonumber \Omega_1^\intercal &= \left[\omega^\intercal \quad
u_2+2\lambda\omega_{1_2}\right] \\
\Omega_2^\intercal &= \left[\omega^\intercal\right] \\
\omega &= 2\lambda\lambda_f^2\omega_2 -
(\lambda^2+2\lambda\lambda_f)y + \lambda^2r \\
\omega_f &= L^{-1}(s)\omega \\
\nonumber \zeta_1^\intercal &= \omega_f^{\intercal}\left[I \quad \Theta_2\right]
\\
\zeta_2^\intercal &= \omega_f^\intercal \\
\dot{\Theta} &= -\text{sign}(D)\Gamma\zeta e
\end{align}

A implementa��o segue abaixo:

\lstinputlisting[style=myMatlab]{../matlab/1/mrac.m} 

\subsection{Caso onde todos os par�metros s�o desconhecidos}
\lstinputlisting[style=myMatlab]{../matlab/2/mrac.m}